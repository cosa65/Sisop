\documentclass[a4paper]{article}
\usepackage[spanish]{babel}
\usepackage[utf8]{inputenc}
%\usepackage{charter}   % tipografia
\usepackage{graphicx}
\usepackage{makeidx}

%\usepackage{float}
\usepackage{amsmath, amsthm, amssymb}
%\usepackage{amsfonts}
%\usepackage{sectsty}
%\usepackage{wrapfig}
\usepackage{listings}
%\lstset{language=C}

\input{codesnippet}
\input{page.layout}
% \setcounter{secnumdepth}{2}
\usepackage{underscore}
\usepackage{caratula}
\usepackage{url}

\usepackage{a4wide}
\usepackage{amsmath}
\usepackage{amsfonts}
\usepackage{graphicx}
\usepackage{color}
\usepackage{todonotes}
%\usepackage[ruled,vlined]{algorithm2e}

\usepackage{float}

\newcommand{\real}{\mathbb{R}}
\newcommand{\nat}{\mathbb{N}}

\newcommand{\atacante}{sanguijuela}
\newcommand{\capitan}{Capit\'an Guybrush Threepwood}
\newcommand{\objeto}{parabrisas}
\newcommand{\nave}{El Pepino Marino}

\newcommand{\revJ}[1]{{\color{red} #1}}


% ******************************************************** %
%              TEMPLATE DE INFORME ORGA2 v0.1              %
% ******************************************************** %
% ******************************************************** %
%                                                          %
% ALGUNOS PAQUETES REQUERIDOS (EN UBUNTU):                 %
% ========================================
%                                                          %
% texlive-latex-base                                       %
% texlive-latex-recommended                                %
% texlive-fonts-recommended                                %
% texlive-latex-extra?                                     %
% texlive-lang-spanish (en ubuntu 13.10)                   %
% ******************************************************** %

%\begin{figure}[h]
  %\begin{center}
	%\includegraphics[scale=0.66]{imagenes/logouba.jpg}
	%\caption{Descripcion de la figura}
	%\label{nombreparareferenciar}
  %\end{center}
%\end{figure}


%\paragraph{\textbf{Titulo del parrafo} } Bla bla bla bla.
%Esto se muestra en la figura~\ref{nombreparareferenciar}.

%\begin{codesnippet}
%\begin{verbatim}

%struct Pepe {

    %...

%};

%\end{verbatim}
%\end{codesnippet}

\begin{document}

\thispagestyle{empty}
\materia{Métodos Numericos}
\submateria{Segundo Cuatrimestre de 2014}
\titulo{Trabajo Práctico II}
\subtitulo{}
\integrante{Nicolás Chamo}{282/13}{nicochamo@hotmail.com}
\integrante{Maximiliano Rey}{}{}
\integrante{Fabricio Previgliano}{430/13}{fjprevi@gmail.com}


\maketitle
\newpage

\thispagestyle{empty}
\vfill
\begin{abstract}

\end{abstract}

\thispagestyle{empty}
\vspace{3cm}
\tableofcontents
\newpage


%\normalsize
\newpage
\section{Ejercicio 1}
\input{ejercicio1}
\newpage
\section{Ejercicio 2}
\input{ejercicio2}
\newpage
\section{Ejercicio 3}
\input{ejercicio3}
\newpage
\section{Ejercicio 4}
\input{ejercicio4}
\newpage
\section{Ejercicio 5}
\input{ejercicio5}
\newpage
\section{Ejercicio 6}
\input{ejercicio6}
\newpage
\section{Ejercicio 7}
\input{ejercicio7}
\newpage
\section{Ejercicio 8}
\input{ejercicio8}
\newpage
\section{Ejercicio 9}
\input{ejercicio9}
\newpage
\section{Ejercicio 10}
\input{ejercicio10}

\end{document}
